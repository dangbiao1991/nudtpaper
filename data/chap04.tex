 \chapter {基于Wi-Fi 信号的手势感知}

 \section{引言与相关工作}



 \section{基于 Wi-Fi 感知的优势及困难}


 \section{系统架构}


 \section{CSI 数据的获取}

 \subsection{定制内核驱动}

 \subsection{数据处理与可视化}

(感知过程需要对数据实时处理,可视化是为了方便研究观察)




\section{CSI 数据变化分析}


\subsection{手势及障碍物对 CSI 幅值的影响}
\subsection{不同样本间的相关性分析}
\subsection{同样本不同子载波间的相关性分析}



\section{感知模型}

\subsection{动作存在性检测}
\subsection{动作数量检测}
\subsection{基于 SVM 的动作分类}
            
SVM 分类算法
分类特征(幅值分布,环境对 CSI 影响的消除)     




\section{性能评估}

\subsection{存在性检测}
\subsection{数量检测}
\subsection{SVM 分类性能}




\section{本章小结}



% \chapter{基于 WiFi 的人体动作感知}
% \section{引言}
% \subsection{感知需求及存在的困难}
% 人体靠近检测

% 动作存在性检测

% 动作数量判断

% \section{已有的工作}
% 基于 RSSI
% 基于 视频
% 基于 红外
% 基于 加速传感器陀螺仪

% \section{基于 CSI 感知人体动作}

% \subsection{理论支持}

% \subsection{实验支持(相关性分析)}
% 正文内容
% \section{系统设计}

% \subsection{CSI 数据获取}
% \subsection{CSI 的预处理}
% 窗口均值 波形变换 滤波 主成分分析

% \subsection{利用 SVM 的动作分类}

% 特征选择及提取过程

% \section{性能评估}
% 正文内容
% \subsection{实验设计}
% \subsection{结果与分析}

% \section{本章小结}