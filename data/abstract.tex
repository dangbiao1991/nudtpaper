\begin{cabstract}
% 随着计算机技术的快速发展,感知数据爆炸式增大。如何智能感知物理世界、如何对感知大数据进行高效计算成为产业界、学术界的热点和难点。本次“智能感知与先进计算高端论坛”特邀了物联网和大数据处理等方向的境内外知名专家学者来湘潭进行主题报告,同时为相关领域专家、师生提供一个前瞻的交流平台。

% 近年来,智能感知技术在多个领域。本文就介绍了两种基于智能终端的被动感知技术--基于声音的人车环境感知技术和基于Wi-Fi信号的手势感知技术。人车环境感知技术意在保护“低头族”的出行安全,减少交通事故。据不完全统计,由“低头族”引起的车祸数量呈逐年上升趋势。为了帮助边走路边使用手机的行人及时发现危险车辆,本文设计了一个基于智能手机麦克风检测车辆靠近的

近年来,智能终端的数量和种类呈现爆发式增长,其上配备的传感器和基础功能也越来越完善。然而随着人们对终端智能程度的要求越来越高,简单的感知应用已不能满足人们的需求。面向特定情景的,更高层次的智能感知开始成为研究的重点。而在智能感知技术中,被动感知技术以其实现成本低,普适性高等特点受到青睐。本文就将介绍两种基于智能终端的被动感知技术。

第一种技术是基于声音的人车环境感知。针对走路使用手机的“低头族”导致的交通事故逐年上升的问题,我们提出了利用智能手机帮助“低头族”感知危险汽车的办法。通过分析手机麦克风获取的音频数据,我们提出了基于时频特征和机器学习的双层感知模型。最后,我们在多种环境下对模型性能进行了评估。结果表明,该模型可以有效感知危险车辆,且对环境有一定鲁棒性。

第二种技术是基于Wi-Fi信号的手势感知。随着Wi-Fi物理层能力的提升,基于高灵敏的CSI感知手势成为一种可行的方法。针对手势感知中遇到手势存在性检测、起始点判断和手势分类三个主要问题,我们提出了一个基于统计特征的感知模型。经实验验证,该模型可以有效识别手势。

\end{cabstract}
\ckeywords{被动感知;普适计算;环境感知;人机交互;安全辅助}

\begin{eabstract}
In recent years, the number and variety of smart devices are increasing explosively. Sensors equipped on smart devices are more plentiful and their functions are already perfect. However, Pepole will not be satisfied with simple sensing applications as their expectations become higher. Smart context-aware sensing will be the focus of research. As a kind of smart sensing technologhy, passive sensing techniques stand out because of its' low cost and ubiquitousness. In this paper,we will introduce two kinds of passive sensing technology based on smart devices.

The first technique is smartphone-based dangerous vehicle detection. As the number of traffic accidents cause by smartphone addicts is increasing every year, we proposed a smartphone based mechanism to help smartphone addicts 
be aware of the approaching vehicle. The microphone of phone is used to collect data. After analysis we design a two level model. The fisrt level utilizes the time frequency features and second level applies the KNN method. The evaluation are performed under different condition. Results show that the model is effecttive to detect approaching car and robust.

The second technique is gesture recognition based on Wi-Fi signal. Because of the extention of physical layer capability, CSI can be achieved now. As CSI is sensitive to small changing of propagation path, it is possible to extract gesture information from CSI. In this paper, we analyse a lot of CSI samples and proposed a statistical feature based classification model. Experimental results show that the model proposed in the paper is effective.

\end{eabstract}
\ekeywords{Passive Sensing;Ubiquitous Computing;Context Awareness;HCI;Safety Assistance}

