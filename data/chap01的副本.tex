\chapter{绪论}

本章的主要内容与学校提供的Word模板中内容一致,图片与表格均采用原始设定大小,%
主要是为了说明格式的统一。%
但是,\LaTeX{}的一些禁则,专业排版的能力,对公式及文献的处理都是得天独厚的,%
我们不必刻意去追求与Word的完美匹配。而且你将会发现,用\LaTeX{}书写论文的美! %

\section{选题背景及其意义}
正文内容

\subsection{(1.1.1 题目)}
正文内容

正文内容

\begin{figure}[htp]
\centering
\includegraphics{picmain}
\caption{图 1.1 名称}
\end{figure}

\subsubsection{(1.1.1.1 题目)}
正文内容

正文内容

正文内容

\subsubsection{(1.1.1.2 题目)}
正文内容

正文内容

正文内容

\subsection{(1.1.2 题目)}
正文内容

正文内容

\begin{figure}[htp]
\centering
\includegraphics{picmain}
\caption{图 1.2 名称}
\end{figure}

\section{国内外研究现状}
正文内容

正文内容

\begin{table}[htp]
\centering
\caption{表 1.2 名称}
\begin{tabular}{|c|c|c|c|c|}
\hline
\makebox[2.07cm][0pt]{} & \makebox[2.07cm][0pt]{} & \makebox[2.07cm][0pt]{} & \makebox[2.07cm][0pt]{} & \makebox[2.07cm][0pt]{} \\
\hline
 & & & & \\
\hline
 & & & & \\
\hline
\end{tabular}
\end{table}

正文内容

正文内容

正文内容

正文内容

\section{论文组织结构}
正文内容

正文内容

正文内容

正文内容

正文内容

正文内容

\subsection{(1.3.1 题目)}
正文内容

\begin{figure}[htp]
\centering
\includegraphics{picmain}
\caption{图 1.3 名称}
\end{figure}

\subsection{(1.3.2 题目)}
正文内容

正文内容

\begin{table}[htp]
\centering
\caption{表 1.2 名称}
\begin{tabular}{|c|c|c|c|c|}
\hline
\makebox[2.07cm][0pt]{} & \makebox[2.07cm][0pt]{} & \makebox[2.07cm][0pt]{} & \makebox[2.07cm][0pt]{} & \makebox[2.07cm][0pt]{} \\
\hline
 & & & & \\
\hline
 & & & & \\
\hline
\end{tabular}
\end{table}

