\chapter{总结与展望}

\section{本文总结}
智能手机、智能手表、智能家居设备等智能终端相继出现,人们对智能生活的想象力似乎才刚刚开始,现有基于传感器感知光线、距离、温度、加速度等基本物理因素的技术已无法满足人们对智能的定义。赋予设备以更加人性的情景感知和交互能力成为迫切的需求。从易用性和普适性的角度出发,基于声音、Wi-Fi信号等数的被动感知技术将会成为研究的新趋势。本文就介绍了两种被动感知技术:基于声音的人车环境感知和基于Wi-Fi信号的手势感知。

基于声音的人车环境感知意在帮助走路玩手机的“低头族”感知突然出现的危险车辆。该技术重点解决三个问题:一,如何判断用户有没有在走路的时候低头玩手机。二,如何判断麦克风收到的声音是不是来自汽车。三,如何判断汽车是不是在向用户靠近。针对问题一,我们通过分析手机加速器数据来判断用户是否在行走,通过分析手机陀螺仪数据来判断用户是否在低头使用手机。针对问题二和问题三,为了在保证识别率并提高响应速度,我们结合了简单时频分析和机器学习方法来解决问题。其中简单的时频分析具有响应速度快,但识别率较低的特点,而所用的机器学习的方法具有识别率高,但响应速度慢的特点。最后,经过实验评估,基于声音的人车环境感知模型,识别率和鲁棒性上均有不错的表现。

基于Wi-Fi信号的手势感知技术意在以一种普适的方式提升设备的智能程度,改善人机交互体验。该技术在实现的过程中主要面临四个问题:一,选取Wi-Fi信号的哪一层数据做数据源。二,如何判断动作的存在与否。三,如何判断动作的起始点。四,如何分辨感知到的是哪个动作。针对问题一,我们经过各方面的比较最终选择了物理层的CSI数据做为数据源,并详细介绍了CSI数据的获取方式。此外还详细分析了CSI数据与环境的关系。针对问题二,我们利用将连续波形转换为方波并结合分析子载波间相关性的方法检测动作的存在。问题三的解决方法是在问题二的基础之上,加上估计相应参数并利用公式推测起始点。针对最后一个问题,我们首先设计了一组包含五个手势的手势集,然后收集了大量实验数据。最后从中提取出了幅值分布特征,并用SVM算法成功分类。实验结果表明,基于Wi-Fi信号的手势感知模型可以有效的工作。


\section{未来展望}

本文的工作取得了阶段性的成果,但其中也存在很多问题等待后续的研究。

人车环境感知方面:智能终端感知到车辆靠近时,人车之间的距离与车速、车型之间的关系还有待进一步的研究。基于声音的感知对环境噪声的容忍上限也需要进一步的评估。能否通过声音来感知用户附近的车流量也将是未来的一个研究方向。除此之外,还可以转换角度,将基于声音的感知应用到司机身上,应用到车联网领域,以带来更安全的驾驶体验。

用Wi-Fi信号感知手势的技术同样存在很多待继续研究的问题。例如,如何提升CSI数据的分析速度,数据采样率对手势感知效果的影响,能否通过降低采样率来相对提升数据分析速度,手势感知是否会影响Wi-Fi的基本通信功能等等。除此之外,基于Wi-Fi信号的感知还可以应用在更多的领域,比如安防,群体行为感知。

智能源自于人们对便捷生活的想象力,人的想象力无限的,相信不久的将来,更多的基于声音,基于无线信号,甚至于基于灯光的感知技术会大量涌现。