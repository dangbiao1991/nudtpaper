\chapter{基于声音的危险车辆靠近检测}
\section{引言与相关工作}

智能手机已然普及,其功能从最初的通讯设备演变为娱乐设备。越来越多人手机不离手,甚至走路时也在低头用手机,被称为“低头族”。湖南某记者于下班高峰时段仅仅统计了 10 分钟就发现不少于 30 人在走路时低头用手机[1],有人更是低头用手机的同时与抢黄灯的车擦肩而过。南京市因行人走路时低头用手机引起的交通事故每月超过 200 起[2]。由“低头族”引起的车祸数量呈逐年上升趋势[3]。行人低头用手机的行为已经严重影响到了交通安全。
解决“低头族”问题的一个方法是通过立法禁止行人低头使用手机,然而执法难度太大,不具有可行性。如果能够为行人提供一种安全辅助,以警告可能发生的危险,则可以大大减少行人低头看手机引起的事故。行人使用手机容易造成交通事故是因为行人在低头使用手机的过程中眼睛和大脑注意力集中在手机屏幕,听觉和反应力变迟钝[4]。因此,如果需要安全辅助,行人手中的智能手机就是最好的选择。作为一个嵌入了丰富传感器的电子设备,智能手机拥有人的五官一样的感知物理世界的能力[5],用它来帮助行人发现危险再合适不过。
智能手机上有多种传感器可以用来感知汽车靠近的危险[6],例如摄像头。腾讯的产品就推出过用摄像头拍摄的实时内容做聊天背景的功能[7]。Tianyu Wang等人也开发了用摄像头检测车辆靠近手机应用 WalkSafe[8]。然而,此种实现方式不实用。一方面,行人在走路时摄像头是对准的地面,无法感知前后左右的来车。另一方面,摄像头耗电量大[9],在夜间需要开启闪光灯才能使用。最后,人在运动过程中拍摄的内容也会不停颤抖,使用户产生眩晕感,用户体验差。
利用智能手机为行人提供安全辅助并非本文首次提出。Trisha Datta 等人就研究了在城区范围内如何通过手机预测行人可能遇到的潜在危险的方法[20]。他们通过分析用户手机的加速传感器数据,结合城市地图数据,来记录并预测行人的路线,推测出行人可能存在的闯红灯过马路等危险行为。
关于利用声音感知汽车行驶状况的方法也有相关研究。Sergei Astapov 等人通过分析汽车噪声数据实现了车辆经过的检测和分类[21]。汽车经过分为靠近和远离的过程,在这个变化过程中,汽车噪声的频谱能量会出现逐渐增大和逐渐减小的过程。作者通过检测这个变化过程来确定是否有车通过。并从汽车噪声信号提取出频谱特征,用机器学习的方法将其分为A、B两类。该方法需等车辆经过之后才能检测到事件,无法用于危险感知。


\section{基于声音检测的优势及困难}

相比于摄像头,使用麦克风感知汽车靠近的危险是一种更加可行的方式。汽车在行驶的过程中会发出各种噪声[10,11],这为用声音感知汽车靠近提供了理论基础。这种方式存在诸多优点:1)当行人的注意力被吸引到手机屏幕,从而对这些声音反应迟钝时,手机的麦克风依然对声音敏感。2)麦克风收集到的声音数据来自于空间的各个方向,其感知空间要远大于摄像头。3)整个感知过程全部后台完成,不存在让用户产生眩晕的情况。4)手机麦克风感知到的声音频谱[12]宽于人类耳朵所能听到的频谱[13],从这个意义上来说,手机麦克风的感知能力要强于人耳。
利用智能手机麦克风感知危险虽然有很多优势,也存在诸多挑战。1)必须在只依靠手机的情况下,感知用户的行走状态以及所处环境。 2)汽车噪声信号成分复杂[14],从中提取信息十分困难。3)危险检测系统对即时性要求高,而智能手机计算能力有限,复杂算法无法直接应用。






\section{车辆噪声分析}

\subsection{音频处理方法}
本文采用智能手机内置麦克风收集的PCM原始语音数据[22]为数据源,通过对声音数据的预处理和特征分析找到与汽车靠近有关的信息。本小节对文中用到的语音处理方法予以说明。

\subsubsection{改进的音量计算}
音量的客观评价尺度是声音的振幅大小,传统音量计算以分贝(dBm)为单位。事实上,分贝的大小反应了人耳对声音大小的感受,与真实振幅不是线性关系[23]。分贝计算过程包含对数计算,为了减少计算量本文采用简化的音量计算方式。对一段时间长度为,共有 N 个采样点的时域信号,其音量的计算方法为:

\begin{equation}
\label{equ:chap3:volume}
V_{\Delta t}= \frac{\sum_{m=0}^{N-1}\left | x\left ( m \right ) \right |}{N}
\end{equation}

\subsubsection{快速傅里叶变换}
音频的频域特征是由时域上的信号分解得到,这一分解过程即是傅里叶变换(Fourier Transform, FT)。对于时域上离散的音频信号,采用离散傅里叶变换进行分解(Discrete Fourier Transform,DFT)。对于一段有限的离散信号 $x(m)$,其长度为 N,则 DFT 公式为:




在该公式中有两个整型变量:m 和 k。其计算复杂性为,为了减小计算复杂性,快速傅里叶变换(Fast Fourier Transforms, FFT)应运而生。本文中使用的 FFT 版本为 MIT 研究员实现的 FFTW[24]。

\subsubsection{滤波}

汽车噪声的成分复杂,在提取其周期性的过程中,本文用Savitzky Golay (SG) 滤波器对信号进行了滤波。SG 滤波器被广泛地运用于数据的平滑除噪,是一种在时域上基于局域多项式最小二乘法拟合的滤波方法。该滤波器最大的特点在于其滤除噪声的同时可以保持信号的形状和宽度不变。

\subsection{汽笛声的检测}

\subsection{车辆噪声音量变化属性}

\subsection{车辆噪声频谱}

\subsection{车辆噪声周期性提取}

周期的提取过程





\section{分类模型}

\subsection{KNN 算法原理}

\subsection{分类特征提取}

\subsection{分类效果}





\section{系统设计}

\subsection{系统架构}

\subsubsection{数据获取}

\subsection{用户行走状态分析模块}

\subsection{噪声属性分析与分类模块}





\section{性能评估}

\subsection{模块性能评估}
\subsection{环境对性能的影响}



\section{本章小结}



% \subsection{感知需求及目标}

% 用户周围有无车辆

% 车辆行驶状态

% \section{现有工作(基于摄像头的)}

% \section{声音感知车辆原理}
% 正文内容

% \section{系统设计}
% 正文内容
% \subsection{音频数据获取}

% \subsection{音频信号的处理方法}
% 正文内容
% \subsection{车辆噪声的时频特征分析}

% \subsection{基于机器学习的噪声来源分类}
% 正文内容

% \section{性能评估}
% 正文内容
% \subsection{实验设计}
% \subsection{结果与分析}

% \section{本章小结}