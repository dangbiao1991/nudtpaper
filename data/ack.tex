
%%% Local Variables:
%%% mode: latex
%%% TeX-master: "../main"
%%% End:

\begin{ack}
  硕士两年半时间过的很快,我拉着行李箱走出长沙火车站的情景仿佛就发生在昨天。虽然心里对毕业后的新生活有一丝小期盼,但更多的是对学校生活的恋恋不舍,对王老师、吕老师和同门师兄弟的不舍。读研这两年我经历了一些小坎坷,若不是师门对我的宽容和帮助,我怕是会过的异常艰难。论文完成之际,我有太多的感激之情想要表达,苦于不能将其全部落于纸面。

  首先,要由衷感谢导师王晓东老师和吕绍和老师的耐心指导。常常庆幸自己当初选择做王老师的学生,王老师在科研上耐心点拨,工作中言传身教,生活中悉心关照,乃良师益友之典范。吕老师在科研上独到的见解,学术上严格的要求,是我论文的推进力,没有吕老师的监督与指导,我的相关论文怕是早已难产。

  其次,要感谢我的父母。感谢父母对我在外求学的支持和理解。父母常常打电话提醒我注意身体,而我对父母主动的关心如此之少,深感愧疚。

  当然,要感谢团结和谐的同门师兄弟,感谢UPCOM。感谢鲁勇师兄对我硕士课题的帮助,帮助我做实验,帮我解答了无数个科研问题,即使出国了也不忘帮我修改论文提纲。感谢汪东师兄在硕士时间对我的关照,使我快速的融入了这里的校园生活,了解了很多办事流程。感谢张闽师兄、杨雪师姐在校期间的照顾,代领我进入了无线网络研究的方向。感谢于彦秋、陈飞在论文开题预审等流程上的帮助,为论文的撰写节约了宝贵的时间。感谢师弟孙学仁帮我整理了参考文献。还要感谢常涛、黄聪、曾祥芳、胡鑫、陈钦届为UPCOM创造了一个和谐愉快的科研生活环境。

  最后还要感谢学员八队的政委朱涛和队长孙友佳,以学生的角度出发,用一种高效人性化的管理方式为我们创造了一个舒适的生活环境。感谢四班的所有同学,没有你们,硕士生活不会过的如此快乐。

  感谢所有在我硕士期间生活上,科研上帮助过我,但又无法在这里一一列出名字的所有人。

\end{ack}
