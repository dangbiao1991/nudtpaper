\chapter{人机交互趋势与相关工作}

\section{人机交互趋势}

(人主动向机器输入,智能设备也会主动向人输出,交互方式趋向于无线化,比如有线耳机变为无线耳机,键盘输入变为语音输入,即双向都趋向于无线)

\section{基于无线的情境感知}

\section{相关工作}


\subsection{基于摄像头的车辆靠近检测}

手势识别,基于摄像头的汽车靠近检测,安防(环境感知)

\subsection{基于信号强度的无线行为识别}
xbox,Leap motion

\subsection{基于 Wi-Fi 信号的人体动作感知}

谷歌雷达的交互,华盛顿大学基于雷达的原理的交互

\subsection{基于无线感知的优势}























% \subsection{需求及要达到的目标}
% \subsubsection{用户是否在行走}
% \subsubsection{用户是否在使用手机}
% \section{现有的工作}

% \section{手机传感器工作原理}
% 正文内容
% \subsection{陀螺仪工作原理}
% 正文内容
% \subsection{加速传感器工作原理}
% 正文内容
% \subsection{距离传感器工作原理}
% 正文内容

% \section{系统设计}
% 正文内容
% 正文内容
% \subsection{手机传感器数据获取}

% \subsection{应用隐马尔可夫模型分析用户行走状态}
% 正文内容
% \subsection{数据融合与决策}

% \section{性能评估}
% 正文内容
% \subsection{实验设计}

% \subsection{结果与分析}

% \section{本章小结}

正文内容
引用示例:书的\upcite{tex, companion},
还有这些\upcite{Krasnogor2004e, clzs, zjsw},关于杂志的\upcite{ELIDRISSI94,
  MELLINGER96, SHELL02},硕士论文\upcite{zhubajie, metamori2004},博士论文
\upcite{shaheshang, FistSystem01},标准文件\upcite{IEEE-1363},会议论文\upcite{DPMG,kocher99},%
技术报告\upcite{NPB2}。中文参考文献\upcite{cnarticle}\textsf{特别注意}





