\chapter{绪论}

\section{选题背景及其意义}
如今,我们生活在一个数字化的世界里。新型计算机技术的应用出现了井喷式的增长,周围的电子设备越来越多,功能越来越强大。智能手机、智能家居、虚拟现实设备\upcite{burdea2003virtual}、自动驾驶汽车\upcite{bilger2013auto}这些十几年前只出现在科幻电影里的技术逐一被实现,极大的方便和丰富了我们的生活。

十年前,物理世界和虚拟世界曾存在清晰的界限,而现在两者间界限已十分模糊。生活中随处可见的智能终端将两个世界紧密联系到了一起。这些带有丰富传感器的智能终端不仅可以实时感知我们的具体位置\upcite{feng2014trac},适时的提供服务,还可以记录我们的生活状态\upcite{chen2015real,del2015tracking,wibisono2013falls},更好的了解自己。例如,当我们去到一个陌生城市,智能手机会向我们推荐周围的酒店和餐馆,从而能更快的融入当地;当我们运动状态不够,或睡眠状态不好,智能手表就是适时的提醒我们注意身体。此外,这些智能终端还可以帮助我们更好的了解世界,甚至创造一个全新的世界。例如,我们可以戴上VR设备体验一个完全虚拟的世界,或者可以戴上微软的Hololens感受一个即非虚拟也非现实的全新世界。 

这些智能终端作为物理和虚拟世界融合的纽带,其智能程度直接决定了人们的生活体验。在完成基本功能的同时,智能表现在能够主动了解用户状态,帮助用户感知环境,节省用户精力;表现在能够主动理解用户意图,提升人机交互体验。比如,当用户边走路边使用手机时,身后突然驶来一辆汽车,手机应该能够及时提醒;当用户做出特定手势时,终端应该能够理解用户意图,并完成指定的功能。

无论是帮助用户感知环境,还是主动理解用户意图,其本质都是一种感知能力,而感知的关键在于对数据的分析。人有五官,眼、耳、口、鼻、手帮我们感知世界,获取信息,大脑对获取的信息分析之后,做出智能的决策。为了帮助智能终端拥有类似人的感知能力,人们发明了摄像头、麦克风、加速计、陀螺仪等各种各样的传感器,从而赋予了智能终端以获取信息的能力。获取信息的能力已具备,对这些信息对分析和处理便成了接下来研究的重点。不同的分析和处理可以赋予设备完全不一样的感知能力。例如,摄像头可以用来拍照,也可以用来做身份识别\upcite{bowyer2006survey};麦克风可以用来简单的获取声音,也可以用来分析人类的睡眠状况\upcite{hao2013isleep};加速计可以用来计步,也可以用来做手势识别\upcite{liu2009uwave}。
处理和分析

从感知设备的角度分类,感知大致可以分为两种方式:一种是需要感知设备主动发射信号,或需要被感知目标配合的,例如,在感知目标身上部署电子标签或者需要目标与感知设备通信,这种方式我们称之为主动感知;另一种是不需要不需要感知设备主动发送信号,也不需要被感知目标的配合的,即不需在感知目标上部署设备,也不需要其与感知设备通信,我们称之被动感知\upcite{rankin2005passive}。解释主动感知与被动感知最好的例子是照相机。在白天光线充足时,物体反射太阳光,照相机照相的过程就是感知物体反射的太阳光,此时为被动感知。当晚上光线不足时,照相机开启闪光灯,照相机感知的是自己发出的光在物体上的反射,此过程就是主动感知。

%有源感知的典型应用是全球定位系统(Global Positioning System,GPS),该系统中目标必须携带GPS芯片,并通过特定协议与卫星通信才能获取目标的位置信息。无源感知的一个典型应用是基于摄像头的车流监控,它不需要在行驶的汽车做任何配合,只需正常行驶即可。

被动感知的更加类似于人类对物理世界的感知方式,更符合智能感知的趋势。我们的感知需求在逐渐扩大,越来越多的目标被纳入我们的感知范围,然而寄希望于每个目标都随着我们的需求更新而升级换代是十分不现实的,不仅成本高,过程也不可控。比如,现在我们想要实时检测路上行驶的汽车速度\upcite{zhou2007moving,leibe2007dynamic},一种方式是为所有车辆升级安装传感器和GPS,另一种方式是用被动感知的方式,比如利用摄像头,声音,无线信号。显然,第二种方式要更加可行。除此之外,在某些领域,被动感知似乎是更优的选择。比如安防,一方面主动发送信号会暴露自身,另一方面我们不可能要求非法传入者能主动配合携带特定设备。%再比如,公共场所不文明行为的检测,目标也不太可能配合我们。

本文,我们将会介绍两种基于智能终端的被动感知技术,一种是基于声音人车环境感知,另一种是基于Wi-Fi信号的手势感知。

\section{主要工作和创新}

本文主要工作集中在基于智能终端的被动感知技术。基于声音人车环境感知主要面向智能手机和智能手表,致力于提升设备智能程度,帮助行人检测正在靠近的危险车辆,从而减少交通事故的发生,保护的行人身安全。基于Wi-Fi信号的手势感知技术面向包括智能手机和智能家居设备在内的所有具有Wi-Fi功能的智能终端,致力于提升终端对用户行为的认知能力,创造一种更加自然流畅的人机交互体验。

\subsection{基于声音的人车环境感知}

随着智能手机的功能日趋丰富,越来越多人手机不离手,纷纷加入“低头族”。即使在马路上,也可看到不少人在低头使用手机。由于注意力集中在手机上,如果遇到突然驶来的车辆,很容易发生交通事故。如何帮助用户感知危险车辆,减少交通事故的发生成为一个亟待解决的问题。

本文从实际需求出发,创新性地提出了利用智能手机麦克风采集声音,感知危险车辆的方法。所做的主要工作有:
\begin{compactenum}
\item 从成分复杂的汽车噪音中提取出了能表征汽车靠近的周期特性。
\item 分析了汽车噪音成份,并拟合出汽车噪音频谱的特征曲线。
\item 提取了有效的瞬时特征,并用机器学习的方法实现了对样本的自动分类。
\item 进行了大量实验对文中提出的方法进行了评估和验证。
\end{compactenum}

\subsection{基于Wi-Fi信号的手势感知}
手势感知已经成为人机交互的一种新趋势。在传统PC上,用户可以使用鼠标键盘与计算机交互。但在新型智能设备上,由于无法接入过多外设,人与设备之间的交互十分困难。在人与VR设备,智能设备的交互问题上,手势是一个很不错的选择。当不在需要双手握着两个重重的VR鼠标玩游戏,不再需要用低效的遥控控制智能电视,用户才能真正的感受到智能带来的流畅体验。

一方面,现有较成熟的手势感知技术多是基于摄像头,可是在每个智能设备上装两个摄像头来做手势感知是不现实的。另一方面,跟随万物互联的大趋势,Wi-Fi成了智能设备的标配。于是基于Wi-Fi信号做手势感知的想法水到渠成。关于这方面的研究,本文主要做了如下工作:
\begin{compactenum}
\item 基于开源代码,修改并适配了TP-Link WR740的网卡驱动,获取到了CSI信息。
\item 设计了一组手势集,并基于该组手势采集了大量CSI数据。
\item 详细分析了CSI幅值及相关性与手势和环境的关系。
\item 完成了可用于检测手势存在性和手势数量的感知模型和基于机器学习的手势分类模型。
\item 设计实验对模型性能进行了评估和验证。
\end{compactenum}

\section{国内外研究现状}

%无源感知有两种定义。一种解释中“源”字表示能源,即利用不需要能源的设备进行感知,多见于无线传感器网络领域。另一种解释中“源”子代表信号源,即专用信号源的感知,此种技术广泛应用于智能感知领域。本文中,无源指后者。

被动感知是近年来逐渐火热的一个研究方向。一方面由于各种智能终端呈现爆发式的增长,为研究的进行提供了设备基础。另一方面,各种传感器相继被发明,为研究提供了数据基础。如果把这些数据比做财富,那我们现在处于刚刚暴富,但不知道怎么利用这边财富的阶段。如何从丰富的传感器数据中挖掘出有用的信息,实现一个全新的功能,是一个十分有意思的研究方向。比如,微软亚洲研究院研究的基于麦克风的手势控制系统和微软基于摄像头的体感交互系统Kinect。

细细数来,被动感知已经在很多领域发挥了重要作用。例如,银行中的防盗系统,利用红外感知人的体温,然后自动报警;基于Wi-Fi的RSSI、GSM、FM的室内定位系统\upcite{yang2012locating,chen2012fm};基于声音、摄像头甚至可见光通信网络的手势感知;基于视频的人流车流检测、人物追踪等等。研究人员正在不断地从这些看似平凡的传感器数据发掘新功能,相信将来会有越来越多的让人意想不到的新功能出现。 


\section{论文组织结构}

本文共有五章,其余四章的组织结构如下:

第二章主要介绍本课题的相关工作和知识,包括移动车辆检测和无线感知的相关研究问题,傅里叶变换,小波分析,IEEE 802.11和OFDM,以及非传感器感知领域的一些工作。

第三章主要介绍汽车靠近检测的需求背景和基于声音的检测实现方法,并详细阐述了方法中用到的模型和提取的时频特征。

第四章分析了基于Wi-Fi信号感知手势的优缺点,介绍了获取CSI的方法,CSI的分析过程以及有效特征的提取过程。并设计实验对结果进行了验证。

第五章对本研究课题的工作做了总结,并展望了未来的工作。

% \chapter{相关工作}

% \section{智能感知}

% \section{情景感知}
% 具体环境感知相关工作
% 摄像头感知等

% \section{无线感知}
% 基于雷达的,多天线阵列的,



